% Beamer Presentation
% LaTeX Template
% Version 1.0 (10/11/12)
%
% This template has been downloaded from:
% http://www.LaTeXTemplates.com
%
% License:
% CC BY-NC-SA 3.0 (http://creativecommons.org/licenses/by-nc-sa/3.0/)
%
%%%%%%%%%%%%%%%%%%%%%%%%%%%%%%%%%%%%%%%%%

%----------------------------------------------------------------------------------------
%	PACKAGES AND THEMES
%----------------------------------------------------------------------------------------


\documentclass[xcolor=dvipsnames]{beamer}

\mode<presentation> {

% The Beamer class comes with a number of default slide themes
% which change the colors and layouts of slides. Below this is a list
% of all the themes, uncomment each in turn to see what they look like.

%\usetheme{default}
%\usetheme{AnnArbor}
%\usetheme{Antibes}
%\usetheme{Bergen}
%\usetheme{Berkeley}
%\usetheme{Berlin}
%\usetheme{Boadilla}
%\usetheme{CambridgeUS}
%\usetheme{Copenhagen}
%\usetheme{Darmstadt}
%\usetheme{Dresden}
%\usetheme{Frankfurt}
%\usetheme{Goettingen}
%\usetheme{Hannover}
%\usetheme{Ilmenau}
%\usetheme{JuanLesPins}
%\usetheme{Luebeck}
%\usetheme{Madrid}
%\usetheme{Malmoe}
%\usetheme{Marburg}
%\usetheme{Montpellier}
\usetheme{PaloAlto}
%\usetheme{Pittsburgh}
%\usetheme{Rochester}
%\usetheme{Singapore}
%\usetheme{Szeged}
%\usetheme{Warsaw}

% As well as themes, the Beamer class has a number of color themes
% for any slide theme. Uncomment each of these in turn to see how it
% changes the colors of your current slide theme.

%\usecolortheme{albatross}
%%\usecolortheme{beaver}
%\usecolortheme{beetle}
%\usecolortheme{crane}
%%\usecolortheme{dolphin}
\usecolortheme{dove}
%\usecolortheme{fly}
%\usecolortheme{lily}
%\usecolortheme{orchid}
%\usecolortheme{rose}
%\usecolortheme{seagull}
%\usecolortheme{seahorse}
%\usecolortheme{whale}
%\usecolortheme{wolverine}

%\setbeamertemplate{footline} % To remove the footer line in all slides uncomment this line
%\setbeamertemplate{footline}[page number] % To replace the footer line in all slides with a simple slide count uncomment this line

%\setbeamertemplate{navigation symbols}{} % To remove the navigation symbols from the bottom of all slides uncomment this line
}

\usepackage[spanish]{babel}
\usepackage[utf8]{inputenc}

\usepackage{graphicx} % Allows including images
\usepackage{booktabs} % Allows the use of \toprule, \midrule and \bottomrule in tables
\usepackage{multimedia}
\usepackage{wrapfig}
\usepackage{multicol}
\setbeamersize{text margin left=0.5mm} 

\setbeamercolor{frametitle}{bg=lightgray}
\setbeamercolor{logo}{bg=gray}
\setbeamercolor{section in sidebar}{fg=black} 
\setbeamercolor{section in sidebar shaded}{fg=gray}
\setbeamercolor{title}{fg=Black}
\setbeamercolor{author in slidebar}{fg=black}
\newcommand{\pu}{\partial^\mu}
\newcommand{\pd}{\partial_\mu}
\newcommand{\pun}{\partial^\nu}
\newcommand{\pdn}{\partial_\nu}
\newcommand{\Aun}{A^\nu}
\newcommand{\Adn}{A_\nu}
\newcommand{\Au}{A^\mu}
\newcommand{\Ad}{A_\mu}
\newcommand{\Bun}{B^\nu}
\newcommand{\Bdn}{B_\nu}
\newcommand{\Bu}{B^\mu}
\newcommand{\Bd}{B_\mu}


%----------------------------------------------------------------------------------------
%	TITLE PAGE
%----------------------------------------------------------------------------------------

\title[Polarización]{Polarización de Partículas Masivas \\ y sin Masa (Fotón)} % The short title appears at the bottom of every slide, the full title is only on the title page

\author[JaimeDGP]{Jaime Díez González-Pardo } % Your name

\institute[UC]{Universidad de Cantabria}

\date{ \today} % Date, can be changed to a custom date

	\begin{document}
		\centering
	
		\begin{frame}
			\titlepage % Print the title page as the first slides
		\end{frame}
		\section{Inv. Gauge}
			\begin{frame}
				\frametitle{Invariancia Gauge}

				\only<1->{
					\begin{equation*}
						\textrm{Cuadrivector potencial eléctrico } \Rightarrow \boxed{\Au = (\phi, \mathbf{A})}
					\end{equation*}
					\begin{equation*}
						\begin{matrix}
							\mathcal{L}_0 = - \frac{1}{4} F^{\mu\nu}F_{\mu\nu} & \Rightarrow & \pd\pu\Aun - \pun\pd\Au = j^\nu 
						\end{matrix}
					\end{equation*}}

				\only<2->{\begin{equation*}
						\textrm{Invarianza Gauge } \Rightarrow \left \{ \begin{matrix}
							\boxed{\Ad \rightarrow \Ad' = \Ad - \pd\chi}
							\\
							\pu\Ad' = \pu\Ad - \Box \chi
							\\
							\begin{matrix}
								\Box \chi = \pu\Ad & | & \pu \Ad' = 0
							\end{matrix}
						\end{matrix}\right .
					\end{equation*}}

				\only<2->{\begin{equation*}
					\begin{matrix}
						\textrm{Operador D'Alembertiano } & \Rightarrow & 
						\begin{bmatrix}
							\Box \equiv g^{\mu\nu}\partial_\mu\partial_\nu = \partial_\mu\partial^\mu
						\end{bmatrix}
					\end{matrix}
				\end{equation*}}

				\only<3->{\hrule}

				\only<3->{\begin{equation*}
						\begin{matrix}
							\textrm{Lorenz gauge condition} & & & & \textrm{Lorenz gauge}
							\\
							\\
							\boxed{\pu\Ad = 0} & & & & \boxed{\Box \Au \equiv \pdn\pun\Au = j^\mu}
						\end{matrix}
					\end{equation*}
				}
			\end{frame}

		\section{El Fotón}

			\begin{frame}
				\frametitle{Polarizaci\'on del Fot\'on}

				\only<1->{Se estudia el fot\'on en ausencia de carga $j^\mu = 0$

				\vspace{-4mm}
					\begin{equation*}
						\Box \Au = j^\mu = 0
					\end{equation*}}

				\only<2->{Se obtiene una ecuaci\'on de ondas con soluci\'on: $\Au = \epsilon^\mu(q) e^{-iqx}$

					\begin{equation*}
						\begin{matrix}
							\Box \Au = \Box (\epsilon^\mu e^{-iqx}) = -q^2\epsilon^\mu e^{-iqx}=0 & \Rightarrow & q^2 = m^2 = 0
						\end{matrix}
					\end{equation*}}

				\only<3->{Aplicando la condici\'on gauge de Lorenz

					\begin{equation*}
						\begin{matrix}
							\pd\Au = \pd(\epsilon^\mu(q) e^{-iqx}) = -iq_\mu\epsilon^\mu e^{-iqx} & \Rightarrow  \boxed{q_\mu\epsilon^\mu = 0}
							\\
							& \downarrow
							\\
							&  \textrm{3 grados de libertad}
						\end{matrix}
					\end{equation*}}
			\end{frame}

			\begin{frame}
				\frametitle{Polarizaci\'on del Fot\'on}

				\only<1->{Imponiendo la condici\'on de Lorenz} \vspace{-4mm}

					\only<1->{\begin{equation*}
						\begin{matrix}
						\Ad \rightarrow \Ad' = \Ad - \pd\Lambda(x) & | & \pu\Ad' = \pu\Ad - \Box\Lambda(x)
						\end{matrix}
					\end{equation*}}

				\only<2->{Se escoge un $\Lambda(x)$ tal que $\Box \Lambda(x) = 0$}

					\only<2->{\begin{equation*}
						\begin{matrix}
							\Lambda(x) = -iae^{-iqx} & \Rightarrow & \Box \Lambda = -q^2 \Lambda
						\end{matrix}
					\end{equation*}}

				\only<3->{\hrule}

					\only<3->{\begin{equation*}
						\begin{matrix}
							\Ad \rightarrow \Ad' = \Ad = \pd \Lambda(x) & = & \epsilon_\mu e^{-iqx}+ia\pd e^{-iqx}
							\\
							\\
							& = & \epsilon_\mu e^{-iqx}+ia (-iq_\mu) e^{-iqx}
							\\
							\\
							& = & (\epsilon_\mu+aq_\mu) e^{-iqx}
						\end{matrix}
					\end{equation*}}
			\end{frame}

			\begin{frame}
				\frametitle{Polarizaci\'on del Fot\'on}
					\vspace{-6mm}

					\only<1->{\begin{equation*}
						\epsilon_\mu \rightarrow \epsilon'_\mu = \epsilon_\mu + aq_\mu
					\end{equation*}
				
					Cualquier vector polarizaci\'on que sea multiplo del cuadrimomento del fot\'on corresponde al mismo fot\'on f\'isico. De esta forma se escoge un $a$ de tal forma que la componente temporal se anule. De esta forma la expresi\'on de la condici\'on gauge de lorenz queda como:

						\begin{equation*}
							q_\mu\epsilon^\mu = 0 \Rightarrow \boxed{\boldsymbol{q\cdot\epsilon} = 0}
						\end{equation*}}

					\only<2->{	\begin{equation*}
						\begin{matrix}
							\boxed{
							\begin{matrix}
								\epsilon_1^\mu = (0,1,0,0) & \textrm{ y } &\epsilon_2^\mu = (0,0,1,0)
							\end{matrix}
							}
							\\ \\
							\boxed{
							\begin{matrix}
								\epsilon_-^\mu = \frac{1}{\sqrt{2}} (0,1,-i,0) & \textrm{ y } &\epsilon_+^\mu = -\frac{1}{\sqrt{2}} (0,1,i,0)
							\end{matrix}
							}
						\end{matrix}
						\end{equation*}}

			\end{frame}

		\section{Part. Masivas}

			\begin{frame}
				\frametitle{Polarizaci\'on de Particulas Masivas}
				\vspace{-4mm}

				\only<1->{\begin{equation*}
					\begin{matrix}
						\textrm{Lagrangiano} & \Rightarrow & \mathcal{L}_m = -\frac{1}{4} F^{\nu\mu}F_{\nu\mu} + \frac{1}{2} m^2 B^\mu B_\mu
						\\
						\\
						\textrm{Euler-Lagrange} & \Rightarrow & (\Box + m^2)\Bu-\pu(\pdn\Bun) = 0
					\end{matrix}
				\end{equation*}}

				\only<2->{\hrule

				\begin{equation*}
					\begin{matrix}
						\pd \cdot \left\{(\Box + m^2)\Bu-\pu(\pdn\Bun)\right\} = 0 \\ \\
						(\Box + m^2)\pd\Bu-\pd\pu(\pdn\Bun) = 0	\\ \\
						(\Box + m^2)\pd\Bu-\Box(\pdn\Bun) = 0 \\ \\
						m^2\pd\Bu = 0 \\
					\begin{matrix}
						\Downarrow & \Downarrow \\
						\boxed{\pd\Bu = 0} & \boxed{(\Box + m^2)\Bu = 0}
					\end{matrix}
					\end{matrix}
				\end{equation*}}
			\end{frame}

			\begin{frame}
				\frametitle{Polarizaci\'on de Particulas Masivas}

				\only<1->{Puesto que para part\'icular masivas se tiene que $q^2 = m^2$ se puede utilizar soluci\'on de onda plana.

					\begin{equation*}
					\begin{matrix}
						\Bu = \epsilon^\mu e^{-iqx} \\ \\			
						\begin{matrix}
							\Box e^{-iqx} = -q^2e^{-iqx} = -m^2e^{-iqx} & \Rightarrow & (\Box + m^2)\Bu = 0
						\end{matrix}
					\end{matrix}
					\end{equation*}}

				\only<2->{Se aplica la condici\'on de lorenz.

					\begin{equation*}
						\pd\Bu = \pd(\epsilon^\mu e^{-iqx}) = -i\epsilon^\mu q_\mu e^{-iqx}
					\end{equation*}}

					\only<3->{\begin{equation*}
						\boxed{
						\begin{matrix}
							& & & &\\
							& &\epsilon^\mu q_\mu = 0 & & \\
							& & \textrm{3 grados de libertad} & &\\
							& & & &\\
						\end{matrix}}
					\end{equation*}}
			\end{frame}

			\begin{frame}
				\frametitle{Polarizaci\'on de Particulas Masivas}

				\only<1->{Si intentamos realizar una transformaci\'on guage:

					\begin{equation*}
						\begin{matrix}
							\Bd \rightarrow \Bd' = \Bd - \pd\chi(x) 
							\\ \\
							\pu\Bd' = \pu\Bd - \Box\chi(x) 
						\end{matrix}
					\end{equation*}}

				\only<2->{ Al contrario que en el caso del fot\'on, $\chi = e^{-iqx}$ no cumple que $\Box \chi = 0$ por lo que se determina que las part\'iculas masivas tiene 3 estados independientes de polarizaci\'on.}

				\only<3->{	\begin{equation*}
						\begin{matrix}
							\boxed{
							\begin{matrix}
								\epsilon_-^\mu = \frac{1}{\sqrt{2}} (0,1,-i,0) & \textrm{ y } &\epsilon_+^\mu = -\frac{1}{\sqrt{2}} (0,1,i,0)
							\end{matrix}
							} \\ \\
							\begin{matrix}
								\epsilon^\mu_L \propto (\alpha, 0, 0, \beta) &\Rightarrow&
							\boxed{\epsilon^\mu_L = \frac{1}{m}(p_z, 0, 0, E)}
							\end{matrix}
						\end{matrix}
						\end{equation*}}
			\end{frame}

		\section{Resumen} 

			\begin{frame}
				\frametitle{Resumen}
					%\centering

					\begin{multicols}{2}
						\only<1->{\begin{equation*}
								\mathcal{L}_0 = -\frac{1}{4} F^{\nu\mu}F_{\nu\mu}
							\end{equation*}}
						\only<2->{\begin{equation*}
								\Box \Au - \pu (\pdn \Aun) = 0
							\end{equation*}}
						\only<3->{\begin{equation*}
							\begin{matrix}
								\pd\Au = 0 &\Rightarrow& \Box \Au = 0
							\end{matrix}
							\end{equation*}}

						\only<1->{\begin{equation*}
								\mathcal{L}_m = -\frac{1}{4} F^{\nu\mu}F_{\nu\mu} + \frac{1}{2} m^2 B^\mu B_\mu
							\end{equation*}}
						\only<2->{\begin{equation*}
								(\Box + m^2) \Bu - \pu (\pdn \Bun) = 0
							\end{equation*}}
						\only<3->{\begin{equation*}
								\begin{matrix}
									\pd\Bu = 0 &\Rightarrow& (\Box + m^2)\Bu = 0
								\end{matrix}
							\end{equation*}}
					\end{multicols}

					\only<4->{\hrule}

					\vspace{-5mm}

					\begin{multicols}{2}
						\only<4->{\begin{equation*}
								\Au = \epsilon(q)e^{-iqx}
							\end{equation*}}
						\only<5->{\begin{equation*}
								\pd\Au = \pd(\epsilon e^{-iqx}) = 0
							\end{equation*}}
						
						\only<4->{\begin{equation*}
								\Bu = \epsilon(q) e^{-iqx}
							\end{equation*}}
						\only<5->{\begin{equation*}
								\pd\Bu = \pd(\epsilon e^{-iqx}) = 0
							\end{equation*}}
					\end{multicols}

					\only<6->{\begin{equation*}
							\boxed{\epsilon^\mu q_\mu = 0} \Rightarrow \textrm{3 grados de libertad de } \epsilon
						\end{equation*}}
			\end{frame}

			\begin{frame}
				\frametitle{Resumen}

					\vspace{-4mm}

					\only<1->{\begin{equation*}
							\boxed{\epsilon^\mu q_\mu = 0} \Rightarrow \textrm{3 grados de libertad de } \epsilon
						\end{equation*}}


					\only<2->{\hrule}

					%\vspace{-5mm}

						\only<2->{\begin{equation*}
								\begin{matrix}
									\pu \Ad' = \pu \Ad - \Box \Lambda(x) & & \pu \Bd' = \pu \Bd - \Box \chi(x)
									\\ \\
									\Box \Lambda = -q^2\Lambda = 0 & & \Box \chi = -q^2\chi = -m^2\chi \neq 0
									\\ \\
									\Lambda(x) = -iae^{-iqx}
								\end{matrix}
							\end{equation*}}

					\only<3->{\hrule}

					\only<3->{\begin{equation*}
						\Ad \rightarrow \Ad' = \Ad - \pd \Lambda(x) = \epsilon_\mu e^{-iqx}+ia\pd e^{-iqx}
					\end{equation*}}

					\only<4->{\begin{equation*}
							\begin{matrix}
								\epsilon_\mu \rightarrow \epsilon_\mu' = \epsilon_\mu + aq_\mu & \Rightarrow &
								\boxed{\begin{matrix}
								 	\boxed{\boldsymbol{\epsilon \cdot q} = 0}
								 	\\
								 	\\
								 	\textrm{2 grados de libertad}
								\end{matrix}}
							\end{matrix}
						\end{equation*}}
			\end{frame}

			\nocite{Thomson:2013zua}

%----------------------------------------------------------------------------
%     BIBLIOGRAPHY
%----------------------------------------------------------------------------


	\bibliographystyle{unsrt}
	\bibliography{biblio}

\end{document} 